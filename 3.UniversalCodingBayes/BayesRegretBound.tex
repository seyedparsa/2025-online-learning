% $Header: /data/cvsroot/Courses/OnlineLearning/talks/talk4/talk4.tex,v 1.4 2006/01/24 08:08:47 yfreund Exp $

\documentclass{beamer}
%\documentclass[handout]{beamer}
% This file is a solution template for:

% - Giving a talk on some subject.
% - The talk is between 15min and 45min long.
% - Style is ornate.

% Copyright 2004 by Till Tantau <tantau@users.sourceforge.net>.
%
% In principle, this file can be redistributed and/or modified under
% the terms of the GNU Public License, version 2.
%
% However, this file is supposed to be a template to be modified
% for your own needs. For this reason, if you use this file as a
% template and not specifically distribute it as part of a another
% package/program, I grant the extra permission to freely copy and
% modify this file as you see fit and even to delete this copyright
% notice. 


\mode<presentation>
{
  \usetheme{Montpellier}

  %\setbeamercovered{transparent}
  % or whatever (possibly just delete it)
}

\usepackage{xmpmulti} % package that defines \multiinclude

\usepackage[english]{babel}

\usepackage[latin1]{inputenc}

\usepackage{times}
\usepackage[T1]{fontenc}
% Or whatever. Note that the encoding and the font should match. If T1
% does not look nice, try deleting the line with the fontenc.

\title[Online Bayes Alg.]% (optional, use only with long paper titles)
{The Online Bayes algorithm}

\author[Freund] % (optional, use only with lots of authors)
{Yoav Freund}
% - Give the names in the same order as the appear in the paper.
% - Use the \inst{?} command only if the authors have different
%   affiliation.

\institute[Universities of Somewhere and Elsewhere] % (optional, but mostly needed)

\subject{Machine Learning}
% This is only inserted into the PDF information catalog. Can be left
% out. 

% If you have a file called "university-logo-filename.xxx", where xxx
% is a graphic format that can be processed by latex or pdflatex,
% resp., then you can add a logo as follows:

% \pgfdeclareimage[height=0.5cm]{university-logo}{university-logo-filename}
% \logo{\pgfuseimage{university-logo}}



% Delete this, if you do not want the table of contents to pop up at
% the beginning of each subsection:
%% \AtBeginSubsection[]
%% {
%%   \begin{frame}<beamer>
%%     \frametitle{Outline}
%%     \tableofcontents[currentsection,currentsubsection]
%%   \end{frame}
%% }


% If you wish to uncover everything in a step-wise fashion, uncomment
% the following command: 

\beamerdefaultoverlayspecification{<+->}

%\newcommand{\R}[1]{{\color{red}{#1}}}
%\newcommand{\vp}{{\mathbf p}}
%\newcommand{\HedgeLoss}{L_{\mbox{\footnotesize Hedge}}}

\newcommand{\newmcommand}[2]{\newcommand{#1}{{\ifmmode {#2}\else\mbox{${#2}$}\fi}}}
\newcommand{\renewmcommand}[2]{\renewcommand{#1}{{\ifmmode {#2}\else\mbox{${#2}$}\fi}}}
\newcommand{\newmcommandi}[2]{\newcommand{#1}[1]{{\ifmmode {#2}\else\mbox{${#2}$}\fi}}}
\newcommand{\newmcommandii}[2]{\newcommand{#1}[2]{{\ifmmode {#2}\else\mbox{${#2}$}\fi}}}
\newcommand{\newmcommandiii}[2]{\newcommand{#1}[3]{{\ifmmode {#2}\else\mbox{${#2}$}\fi}}}

\newcommand{\algfnt}{\bf}

\newmcommand{\ouralg}{{\mbox{\algfnt Hedge}({\eta})}}

\newmcommand{\iter}{T}

\newfont{\cmmib}{cmmib10}
\newcommand{\boldell}{{\mbox{\cmmib \symbol{'140}}}}


\newmcommandi{\costvec}{{\boldell}_{#1}}
\newmcommandii{\cost}{{\ell}^{#1}_{#2}}

\newmcommandi{\rd}{\tilde{#1}}

\newmcommandi{\distvec}{{\bf p}^{#1}}
\newmcommandi{\rddistvec}{\rd{\bf p}^{#1}}
\newmcommandii{\dist}{{p}^{#1}_{#2}}
\newmcommandii{\rddist}{\rd{p}^{#1}_{#2}}

\newmcommandi{\bdistvec}{{\bf q}^{#1}}
\newmcommandii{\bdist}{{q}^{#1}_{#2}}

\newmcommandi{\wtvec}{{\bf w}^{#1}}
\newmcommandi{\rdwtvec}{\rd{\bf w}^{#1}}
\newmcommandii{\wt}{{w}^{#1}_{#2}}
\newmcommandii{\rdwt}{\rd{w}^{#1}_{#2}}


\newcommand{\Nweight}[2]{V_{#1}^{#2}}	%the normalized weight
\newcommand{\dweight}[2]{w^{#2}(#1)} % initial density measure
\newcommand{\TEloss}[1]{L_{#1}}	%total loss of expert i
\newcommand{\BEloss}{L_{\min}}	%total loss of the best expert
\newcommand{\TAloss}{L_A}	%total loss of algorithm
\newcommand{\weight}[2]{W_{#1}^{#2}} % weight assigned to expert
\newcommand{\btheta}{\hat{\theta}}

\newcommand{\R}[1]{{\color{red}{#1}}}
\newcommand{\B}[1]{{\color{blue}{#1}}}
\newcommand{\RM}[1]{{\color{red}{$#1$}}}


%BANDITS
\newcommand{\Aplay}{{\bf Hedge}}
\newcommand{\Aest}{{\bf Exp3}}
\newcommand{\Aesthp}{{\bf Exp3.P}}
\newcommand{\Aestg}{{\bf Exp3.P.1}}
\newcommand{\Aests}{{\bf Exp3.S}}
\newcommand{\Aessg}{{\bf Exp3.S.1}}
\newcommand{\Astrat}{{\bf Exp4}}
\newcommand{\Abound}{{\bf Exp3.1}}
\newcommand{\Gbest}{G_{\rm max}}

\newcommand{\defeq}{\stackrel{\rm def}{=}}
\newcommand{\compl}{\mbox{\sc h}}
\newcommand{\theset}[2]{\{ {#1} \,:\, {#2} \}}

\newmcommandii{\stratv}{\mbox{\boldmath $\xi$}^{#1}({#2})}

%Games paper
\newmcommand{\M}{\bf M}
\newmcommand{\dM}{\M'}
\newmcommand{\Row}{\bf R}
\newmcommand{\dRow}{\R'}
\newmcommand{\C}{\bf C}
\newmcommand{\dC}{\C'}
\newmcommand{\D}{D}
\renewmcommand{\P}{\bf P}
\newmcommand{\Q}{\bf Q}
\newmcommand{\Dt}{\D_t}
\newmcommand{\Pt}{\P_t}
\newmcommand{\Qt}{\Q_t}
\newmcommand{\Pstar}{\P^*}	% the min/max optimal mixed strategy
\newmcommand{\Pref}{\tilde{\P}}	% a reference mixed strategy (not
				% necessarily min/max)
\newmcommand{\Qstar}{\Q^*}
\newmcommand{\Pa}{\overline{\P}}
\newmcommand{\Qa}{\overline{\Q}}
\newmcommand{\Qh}{\hat{\Q}}
\newmcommandi{\trans}{{#1}^{\rm T}}
\newmcommand{\mhx}{\M(h,x)}
\newmcommand{\mxh}{\dM(x,h)}
\newmcommand{\mpq}{\M(\P,\Q)}
\newmcommand{\mpsq}{\M(\Pstar,\Q)}
\newmcommand{\mpsqt}{\M(\Pstar,\Qt)}
\newmcommand{\mptqt}{\M(\Pt,\Qt)}
\newmcommand{\mptt}{\M(\Pt,t)}
\newmcommand{\mptq}{\M(\Pt,\Q)}
\newmcommand{\mpqt}{\M(\P,\Qt)}
\newcommand{\minp}{\min_{\P}}
\newcommand{\maxq}{\max_{\Q}}
\newcommand{\RE}[2]{{\rm RE}\left( {#1} \; \parallel \; {#2} \right) }

\newmcommand{\sumt}{\sum_{t=1}^T}
\newmcommand{\sumin}{\sum_{i=1}^n}
\newmcommand{\delt}{\Delta_{T,n}}
\newcommand{\nextline}{\vspace{0.2cm}\\}   % a little space for equation arrays

\newcommand{\lwalg}{\mbox{\rm MW}}
\newcommand{\lwalgvar}{\mbox{\rm vMW}}

%%
\newcommand{\E}{\mbox{\rm\bf E}}
\newcommand{\p}[2]{p_{#1}(#2)}
\newcommand{\q}[2]{q_{#1}(#2)}
\newcommand{\x}[2]{x_{#1}({#2})}
\newmcommand{\bx}{\mbox{\boldmath$x$}}
\newmcommandi{\xv}{\bx({#1})}
\newmcommand{\xvt}{\xv{t}}
%\newcommand{\w}[2]{w_{#1}({#2})} replaced by \wt, but remember to switch order of parameters i and t
\renewcommand{\i}[1]{i_{#1}}
\newcommand{\hx}[2]{\hat{x}_{#1}(#2)}
\newcommand{\hxit}{\hx{\i{t}}{t}}
\newcommand{\pit}{\p{\i{t}}{t}}
\newcommand{\xit}{\x{\i{t}}{t}}
\newcommand{\expb}[1]{\exp\left(#1\right)}

\newcommand{\vp}{{\mathbf p}}
\newcommand{\vu}{{\mathbf u}}
\newcommand{\vv}{{\mathbf v}}
\newcommand{\vx}{{\mathbf x}}
\newcommand{\vy}{{\mathbf y}}
\newcommand{\vw}{{\mathbf w}}
\newcommand{\vz}{{\mathbf z}}
\newcommand{\vq}{{\mathbf q}}
\newcommand{\vg}{{\mathbf g}}

\newcommand{\vecq}{{\bf q}}
\newcommand{\vecp}{{\bf p}}


\newcommand{\HedgeLoss}{L_{\mbox{\footnotesize Hedge}}}

\newcommand{\W}{\vec{W}}
\newcommand{\V}{\vec{V}}
\newcommand{\X}{\vec{X}}
\newcommand{\vb}{\vec{b}}
%\newcommand{\loss}{\vec{\ell}}
\newcommand{\loss}{L}
\newcommand{\elloss}[2]{\ell_{#2}\left( #1 \right)} %loss of expert i at time t
\newcommand{\lossvec}[1]{{\mathbf \ell}_{#1}}       %loss of expert at time t
\newcommand{\w}[1]{\makebox[12pt]{{#1}}}
\newcommand{\Rps}{\mbox{\tt R}}
\newcommand{\rPs}{\mbox{\tt P}}
\newcommand{\rpS}{\mbox{\tt S}}
\newcommand{\rpstie}{\w{$\frac{1}{2}$}}
\newcommand{\rpswin}{\w{$0$}}
\newcommand{\rpsloss}{\w{$1$}}

\newmcommand{\decspace}{\Delta}
\newmcommand{\decsym}{\delta}
\newmcommandi{\dec}{\decsym^{#1}}
\newmcommand{\decdistsym}{\cal D}
\newmcommandi{\decdist}{{\decdistsym}^{#1}}

\newmcommand{\simpdistspace}{{\bf \cal S}}
\newmcommand{\domset}{{\rm dom}(\decdistsym)}

\newmcommand{\expdistsym}{{\cal E}}
\newmcommandii{\expdist}{{\expdistsym}^{#1}_{#2}}
\newmcommand{\expdecsym}{{\varepsilon}}
\newmcommandii{\expdec}{\expdecsym^{#1}_{#2}}

\newmcommand{\outspace}{\Omega}
\newmcommand{\outsym}{\omega}
\newmcommandi{\out}{\outsym^{#1}}

%\newmcommandii{\Dkl}{D_{\mbox{kl}}\paren{#1||#2}}
\newmcommandii{\Dkl}{{\rm {KL}}\paren{{#1}\;||\;{#2}}}

\newmcommandi{\sumwts}{\sum_{i=1}^N \wt{#1}{i}}

\newmcommand{\lossalg}{L_A}
\newmcommand{\lossouralg}{{L_{\mbox{\scriptsize\algfnt Hedge}(\eta)}}}
\newmcommand{\lossS}{{L_{\mbox{\scriptsize\algfnt S}}}}
\newmcommandi{\lossi}{L_{#1}}
\newmcommandii{\lossit}{L_{#1}^{#2}}

\newmcommandi{\upbnd}{\tilde{#1}}

\newcommand{\angles}[1]{{\left\langle {#1} \right\rangle}}
\newcommand{\paren}[1]{{\left( {#1} \right)}}
\newcommand{\brac}[1]{{\left[ {#1} \right]}}
\newcommand{\braces}[1]{{\left\{ {#1} \right\}}}

\newcommand{\abs}[1]{{\left| {#1} \right|}}
\newcommand{\ceiling}[1]{{\left\lceil {#1} \right\rceil}}

\newfont{\msym}{msbm10}
\newcommand{\real}{\mbox{\msym R}}

\newmcommand{\updatefcn}{U_\eta}

%% \newtheorem{theorem}{Theorem}	
%% \newtheorem{lemma}[theorem]{Lemma}
%% \newtheorem{corollary}[theorem]{Corollary}
%% \newtheorem{definition}{Definition}

%\newcommand{\proof}{\noindent{\bf Proof:} }
%\newcommand{\example}[1]{{\em Example #1.} }
%\newcommand{\qed}{\rule{0.7em}{0.7em}}

\newcommand{\WeakAlg}{\mbox{\algfnt WeakLearn}}
\newcommand{\Boost}{\mbox{\algfnt AdaBoost}}
\newcommand{\EX}{\mbox{\bf EX}}
\newmcommand{\hf}{h_{{f}}}
\newmcommand{\rdhf}{\rd{h}_{{f}}}
\newmcommand{\hfT}{h^T_{{f}}}
\newmcommand{\ranh}{{b}}

\newmcommand{\conclass}{{\cal C}}

\newmcommand{\badvec}{{\bf b}}
\newmcommandi{\bad}{{b}_{#1}}

%%%%%%%% New commands defined for the game-playing paper

\newmcommand{\hedge}{\algfnt Hedge}
\newmcommand{\play}{\algfnt Play}
\newmcommandi{\Glossvec}{{\bg y}^{#1}}
\newmcommandii{\Gloss}{{y}^{#1}_{#2}}
%\newmcommandi{\action}{{I}_{#1}}
\newmcommandi{\Gdistvec}{{\bf \tilde{p}}^{#1}}
\newmcommandii{\Gdist}{{\teilde{p}}^{#1}_{#2}}

%%%%%%%%%%%%%%%%%%%%%%%%%%%%%%%%%%%%%%%%%%%%%%%%%%%%%
\newmcommand{\Idistvec}{{D}}
\newmcommandi{\Idist}{\Idistvec({#1})}
\newmcommand{\Idistt}{\Idistvec_t}

\newmcommand{\Xdist}{{\cal P}}
\newmcommand{\emp}{\hat{\epsilon}}

\newmcommand{\classpc}{Y}
\newmcommand{\numclass}{k}
\newmcommandii{\prob}{\mbox{\rm Pr}_{#1}\left[{#2}\right]}
\newmcommandii{\exval}{\mbox{\rm E}_{#1}\left[{#2}\right]}

%\usepackage{amsmath}
\DeclareMathOperator*{\argmax}{argmax} % thin space, limits underneath in displays
\DeclareMathOperator*{\argmin}{argmin} 

\newcommand{\RR}{{\mathbf R}}
\newcommand{\rr}{{\mathbf r}}
\newcommand{\Btheta}{\bm{\theta}}
\newcommand{\regret}{\mbox{Regret}}

%%% Conditional probabilities
\newmcommandii{\condp}{p\left( #1 \left| #2 \right. \right)}

\newmcommand{\lab}{y}
\newmcommand{\ploss}{\mbox{ploss}}
\newmcommandii{\avploss}{\ploss_{#1}({#2})}
\newcommand{\sfrac}[2]{\mbox{$\frac{#1}{#2}$}}

\newcommand{\mboosta}{\mbox{\algfnt AdaBoost.M1}}
\newcommand{\mboostb}{\mbox{\algfnt AdaBoost.M2}}
\newcommand{\mboostr}{\mbox{\algfnt AdaBoost.R}}

%\newmcommand{\slos}{\mbox{ploss}}
%\newmcommandiii{\sloss}{\slos_{#1}({#2},{#3})}
%\newmcommandiii{\avsloss}{\slos_{{#1},{#2}}({#3})}

\newmcommandii{\vwt}{{W}^{#1}_{#2}}

\newcommand{\figline}{\rule{\textwidth}{1pt}}

%\newmcommandi{\1}{{\bf 1}({#1})}
\newmcommandi{\1}{[\![{#1}]\!]}

\newmcommand{\confcn}{\kappa}
\newmcommandi{\erint}{\abs{\int_{y_i}^{h_t(x_i)} {#1} dy}}
%\newmcommandi{\erint}{\int_{\min\{y_i,h_t(x_i)\}}^{\max\{y_i,h_t(x_i)\}}{#1}dy}

% convex set
\newcommand{\cK}{{\cal K}}
\newcommand{\project}{{\Pi_{\cK}}}

\begin{document}

%\iffalse %%%%%%%%%%%%%%%%%%%%%%%%%%%%%%%%%%%%%%%%%%%%%%%%%%%%%%%%%%%%%%%%%%

\begin{frame}
  \titlepage
\end{frame}

\begin{frame}
  \frametitle{Outline}
  \tableofcontents[pausesections]
  % You might wish to add the option [pausesections]
\end{frame}

\section{Classical Bayesian Statistics}

\begin{frame}
\frametitle{The Bayesian Generative Process}
\begin{itemize}
\item Let \R{$\Theta$} be a set of distrubutions over a space \R{$X$}.
  \newline
  Example: a \R{$d$} dimensional Gaussian
  distribution over \R{$R^d$}. \R{$\theta=(\vec{\mu}, \Sigma)$}
\item
  Let \R{$D$} be the {\bf prior} distribution over \R{$\Theta$}
\item {\bf Selecting Model:} \R{$\theta \in \Theta$} is chosen according to
  the prior \R{$\D$}
\item {\bf Generating Data:} \R{$x_1,x_2,\ldots,x_n$} are generated
  IID according to \R{$\theta$}
\end{itemize}
\end{frame}

\begin{frame}
\frametitle{The Bayes optimal prediction}
\begin{itemize}
\item The {\bf Posterior distribution}: the conditional probability of
  the model \R{$\theta$} given the data \R{$x_1,x_2,\ldots,x_n$}.
\R{$$P(\theta|x_1,x_2,\ldots,x_n) = \frac{1}{Z} \D(\theta) \prod_{i=1}^n P(x_i|\theta)$$}
\item {\bf Posterior average:} predict the distribution of a new
  example \R{$x_{n+1}$} with the conditional probability:
  \R{$$ P(x_{n+1} | x_1,x_2,\ldots,x_n) = \sum_{\theta \in \Theta}
    P(x_{n+1}|\theta) P(\theta|x_1,x_2,\ldots,x_n)$$}
\end{itemize}
\end{frame}

\begin{frame}
\frametitle{In what sense is the posterior average optimal?}
\begin{itemize}
\item It is the optimal perdiction if the data is generated according
  to the Bayesian generative process.
\item What if the data is not generated by any of the models?
\item Classical analysis cannot be used.
\item {\bf We will show} a tight bound on the regret!.
\end{itemize}
\end{frame}


\section{Combining experts in the log loss framework}

\begin{frame}
\frametitle{The log-loss framework}
\begin{itemize}
\item Algorithm \R{$A$} predicts a sequence \R{$c^1,c^2,\ldots, c^T$}
over alphabet \R{$\Sigma = \{1,2,\ldots,k\}$}
\item The prediction for the \R{$c^t$}th is a distribution over \R{$\Sigma$}:\\
\R{$\vp_A^t = \langle p_A^t(1),p_A^t(2),\ldots,p_A^t(k) \rangle$} 
\item When $c^t$ is revealed, the loss we suffer is \R{$-\log p_A^t(c^t)$}
\item The {\color{blue}cumulative log loss}, which we wish to minimize, 
is \R{$\lossalg^T = -\sum_{t=1}^T \log p_A^t(c^t)$}
\item \R{$\lceil \lossalg^T \rceil$} is the code length if \R{$A$} is combined with arithmetic coding.
\end{itemize}
\end{frame}

\begin{frame}
\frametitle{The game}
\begin{itemize}
\item Prediction algorithm \R{$A$} has access to \R{$N$} experts.
\item The following is repeated for \R{$t=1,\ldots,T$}
\begin{itemize}
\item Experts generate predictive distributions: \R{$\vp_1^t,\ldots,\vp_N^t$}
\item Algorithm generates its own prediction \R{$\vp_A^t$}
\item \R{$c^t$} is revealed.
\end{itemize}
\item {\bf Goal:} minimize regret:
\R{\[
-\sum_{t=1}^T \log p_A^t(c^t) + \min_{i=1,\dots,N} \paren{-\sum_{t=1}^T \log p_i^t(c^t)} 
\]}
\end{itemize}
\end{frame}

\section{Review: The online Bayes Algorithm}

\begin{frame}
\frametitle{The online Bayes Algorithm}
\begin{itemize}
\item {\color{blue} Total loss} of expert \R{$i$}
\R{$$L_i^t = - \sum_{s=1}^{t} \log p_i^s(c^s);\;\;\; L_i^0 = 0$$}
\item {\color{blue}Weight} of expert \R{$i$}
\R{$$\wt{t}{i} = \wt{1}{i} e^{-L_i^{t-1}} = \wt{1}{i} \prod_{s=1}^{t-1} p_i^s(c^s)$$}
\item
Freedom to choose initial weights.\\
 \R{$\wt{1}{t} \geq 0$}, \R{$\sum_{i=1}^n \wt{1}{i} = 1$}
\item {\color{blue}Prediction} of algorithm \R{$A$}
\R{\[
\vp_A^t = \frac{\sum_{i=1}^N \wt{t}{i} \vp_i^t}{\sum_{i=1}^N \wt{t}{i}}
\]}
\end{itemize}
\end{frame}

\subsection{Comparison to \ouralg}

\begin{frame}
\frametitle{The \ouralg Algorithm}
Consider action \R{$i$} at time \R{$t$}
\begin{itemize}
\item Total loss:
\R{$$L_i^t = \sum_{s=1}^{t-1} \ell_i^s$$}
\item Weight:
\R{$$\wt{t}{i} = \wt{1}{i} e^{-\eta L_i^t}$$}
Note freedom to choose initial weight (\R{$\wt{1}{i}$})
\R{$\sum_{i=1}^n \wt{1}{i} = 1$}.
\item
\R{$\eta>0$} is the learning rate parameter. Halving: \R{$\eta \to \infty$}
\item Probability:
\R{$$\dist{t}{i} = \frac{\wt{t}{i}}{\sum_{j=1}^N \wt{t}{i}},\;\;
\pause     \distvec{t} = \frac{\wtvec{t}}{\sum_{j=1}^N \wt{t}{i}}$$}
\end{itemize}
\end{frame}

\section{Review: The performance bound}
%\fi %%%%%%%%%%%%%%%%%%%%%%%%%%%%%%%%%%

\begin{frame}
\frametitle{Cumulative loss vs. Final total weight}

\onslide<1-> Total weight: \R{$W^t \doteq \sum_{i=1}^N \wt{t}{i}$}

\onslide<2-> \R{$$
\frac{W^{t+1}}{W^t}  =  \frac{\sum_{i=1}^N \wt{t}{i} e^{\log p_i^t(c^t)}}{\sum_{i=1}^N \wt{t}{i}} 
\onslide<3->          =   \frac{\sum_{i=1}^N \wt{t}{i} p_i^t(c^t)}{\sum_{i=1}^N \wt{t}{i}} 
\onslide<4->        =  p_A^t(c^t)
$$}
\onslide<5-> \R{$$ -\log \frac{W^{t+1}}{W^t} = -\log p_A^t(c^t) $$}
\R{\[
\onslide<8-> -\log W^{T+1} =
\onslide<6-> -\log \frac{W^{T+1}}{W^1} = -\sum_{t=1}^T \log p_A^t(c^t)
\onslide<7-> = L_A^T
\]}
\onslide<9-> \R{\bf EQUALITY} not bound!
\end{frame}

\begin{frame}
\frametitle{Simple Bound}
\begin{itemize}
\item Use uniform initial weights \R{$\wt{1}{i} = 1/N$}
\item Total Weight is at least the weight of the best expert.
\R{\begin{eqnarray*}
L_A^T & = & -\log W^{T+1} 
\pause = -\log \sum_{i=1}^N \wt{T+1}{i} \\
\pause & = & -\log \sum_{i=1}^N \frac{1}{N} e^{-L_i^T} 
\pause  =  \log N - \log \sum_{i=1}^N e^{-L_i^T}\\
\pause & \leq & \log N - \log \max_i e^{-L_i^T}  
\pause = \log N + \min_i L_i^T
\end{eqnarray*}}
\item Dividing by $T$ we get
\R{$ \frac{L_A^T}{T} = \min_i \frac{L_i^T}{T} + \frac{\log N}{T} $}
\end{itemize}
\end{frame}

\subsection{Comparison to \ouralg}

\begin{frame}
\frametitle{Regret bound for \ouralg }
\begin{itemize}
\item Tuning \R{$\eta$} as a function of \R{$T$} (uniform prior).
\item trivially \R{$\min_i \lossi{i} \leq T$}, yielding
\R{\[
\lossouralg \leq \min_i \lossi{i} + \sqrt{2 T \ln N} + \ln N
\]}
\item per iteration we get:
\R{\[
\frac{\lossouralg}{T} \leq \min_i \frac{\lossi{i}}{T} + \sqrt{\frac{2 \ln N}{T}} + \frac{\ln N}{T}
\]}
\item Compare to regret bund for Bayes Algorithm:
\R{$ \frac{L_A^T}{T} = \min_i \frac{L_i^T}{T} + \frac{\log N}{T} $}

\end{itemize}
\end{frame}
\end{document}

